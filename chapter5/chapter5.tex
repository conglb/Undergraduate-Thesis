\mychapter{5}{CHAPTER 5. CONCLUSION AND FUTURE WORKS}\label{ch:chap5} 
This chapter concludes the work and then brings out some potential future works. Also, the aim and objectives of the thesis are reviewed, while the achievements are summarised. \par 

\section{Conclusion}

This thesis was conducted to explore how deep learning models can be used for applications requiring artificial intelligence on smartphones and/or other mobile devices. The minority of current deep learning models is compatible with mobile access. Some of such models are mentioned in the second chapter; the reasons why they are not suitable for deployment on mobile devices are listed as well. Moreover, the difficulties in deployment and supplying real-time experience are comprehensively addressed.
 \par
This thesis began with background information and user cases in order to understand the need and purpose of the work. Afterward, a mobile beauty application was designed and developed as a functional prototype for Android mobile phones. Methods for the important issues of the app are analyzed thoroughly and solved besides. Testing and evaluation were also conducted during and at the end of the thesis project by quantitative assessment.
The mobile application includes camera view, real-time recoloring, color scanning. All of them are the crucial features of beauty applications based on the user study. 
Meanwhile, the mobile application is designed to express consistency with the initial requirement from both functionality and user interface point of view. Moreover, as an application running on smartphones, the design and development also considers the limitation of hardware platform, such as computation power, memory size, etc. \par


\section{Future works}

According to the feedback from testing evaluation, a couple of phenomena should be taken care of, and even modified, such as the model optimization for reducing inference time. And more features could be designed and implemented, such as hairstyle transfer.
\par
More importantly, from the deep learning point of view,  an more effective architecture or base network can also be adapted for the work. In terms of frameworks, both Android and Tensorflow have been releasing out many convenient tools for AI developers, lately are metadata and model binding, which can be adapted to the segmentation application. So adapting those addons to this segmentation application is meaningful and effective. \par

Furthermore, the quality of rendering leaves plenty of room for improvement. As we know, the methods of rendering in the thesis are simply to overlay the mask onto the camera view. For that, color transfer applied directly to the camera's images can be potential. 
\par